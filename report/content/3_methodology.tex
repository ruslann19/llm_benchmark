\newpage

\section{Методология работы бенчмарка}

Данный раздел описывает методологию построения обновляемого бенчмарка для оценки больших языковых моделей. Ключевой принцип заключается в создании не статичного набора данных, а динамического конвейера, способного автоматически извлекать, верифицировать и актуализировать вопросы для тестирования.

\subsection*{Выбор источника данных}

В основе бенчмарка лежит идея использования публичных, регулярно обновляемых источников вопросов. В качестве первичного и концептуального источника была выбрана популярная российская телевизионная викторина \textbf{«Своя игра»}.

\textbf{Обоснование выбора:}
\begin{itemize}
    \item \textbf{Высокое качество контента:} Вопросы викторины составлены профессиональными редакторами и тестируются на живых игроках, что гарантирует их содержательность, корректность и адекватный уровень сложности.
    \item \textbf{Проверяемость:} Для каждого вопроса существует точный, верифицированный ответ, что критически важно для объективной автоматической оценки моделей.
    \item \textbf{Разнообразие тем:} Вопросы охватывают широкий спектр областей знаний: история, наука, искусство, литература, география, спорт и другие, что позволяет оценивать \textit{общие знания} и эрудицию модели.
    \item \textbf{Регулярность и актуальность:} Новые выпуски игры публикуются на официальном форуме в текстовом формате с высокой частотой, обеспечивая постоянный приток \textbf{свежих данных}, которые гарантированно отсутствуют в обучающих наборах моделей, выпущенных ранее.
    \item \textbf{Структурированность:} Текстовые отчёты на форуме сохраняют базовую структуру (темы, стоимость вопросов, ответы), что упрощает их последующий анализ.
\end{itemize}

\textbf{Масштабируемость подхода:} Выбранная методология не ограничивается одним источником. Архитектура конвейера спроектирована с учётом возможности интеграции дополнительных потоков данных:
\begin{itemize}
    \item Другие викторины и интеллектуальные игры (как русскоязычные, так и мультиязычные).
    \item Новостные ленты и дайджесты, позволяющие формулировать вопросы на основе текущих событий.
    \item Специализированные форумы и энциклопедии.
\end{itemize}
Таким образом, бенчмарк эволюционирует из фиксированного набора в \textbf{самообновляемую систему} агрегации оценочных данных из множества источников.
