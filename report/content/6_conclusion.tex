\newpage

\section{Заключение}

В ходе данной научно-исследовательской работы была изучена проблема объективной оценки больших языковых моделей.
Была исследована ограниченность традиционных статических бенчмарков, таких как MMLU.
Основное внимание уделялось современным динамическим подходам, включая LiveBench и Chatbot Arena.
На основе проведённого анализа была сформулирована цель создания нового инструмента.
Целью стала разработка динамического, обновляемого бенчмарка, устойчивого к загрязнению данных.

Была предложена и детально спроектирована архитектура такого бенчмарка.
Ключевым принципом стал автоматизированный сбор вопросов из публичных, постоянно обновляемых источников.
В качестве демонстрационного источника была выбрана викторина «Своя игра».
Для преобразования неструктурированных текстов в данные был применён инновационный метод.
Этот метод использует саму большую языковую модель в качестве универсального и гибкого парсера.
Была разработана модульная система, включающая этапы сбора, структурирования, хранения и оценки.
Для хранения данных была создана реляционная база данных на SQLite с перспективой перехода на PostgreSQL.
Управление системой реализовано через консольное приложение с базовым набором команд.

В рамках экспериментальной части был успешно развёрнут и протестирован конвейер сбора данных.
Парсинг с использованием LLM показал высокую точность.
Были получены первые оценки моделей в двух режимах: через API и веб-интерфейс.
Результаты выявили существенное влияние методологии оценки на итоговые метрики.
Было установлено, что подход LLM-as-a-Judge требует тщательной калибровки для обеспечения консистентности.
Несмотря на это, бенчмарк продемонстрировал способность дифференцировать модели по уровню знаний.

По итогам работы намечены конкретные направления для дальнейшего развития.
\begin{itemize}
    \item Первоочередной задачей является стандартизация и улучшение процедуры автоматического судейства.
          Необходимо интегрировать API ключевых коммерческих и открытых моделей для полной автоматизации тестирования.

    \item Планируется расширить спектр источников данных, добавив новостные ленты и другие викторины.

    \item Для удобства пользователей будет разработан веб-интерфейс с визуализацией результатов и лидербордами.
\end{itemize}

Долгосрочной целью является создание общедоступной, самообновляемой платформы для объективного оценивания LLM.
Таким образом, работа закладывает основу для инструмента, способного эволюционировать вместе с развитием языковых моделей.