%%%    Тип работы
%\LabWork                      %   тип документа -- лабораторная работа
%\LabWorkNo{4}                 %   номер лабораторной работы
%\variant{12}                   %   вариант лабораторной
%\KursWork
%\KursProject
\NIR
%\VKR
%\NKR
%%%  Данные для оформления

%%%  Данные для оформления

\author{Р.Н. Новиков}                     %   автор
\authorfull{Новиков Руслан Николаевич}  %   автор -- Фамилия, Имя, Отчество
\workyear{2025}                         %   год выполнения работы
\title{Обновляемый бенчмарк для оценки общих знаний больших языковых моделей}   %  название
% \discipline{}              %  учебная дисциплина
\group{ФН12-31М}                          %   группа
\faculty{Фундаментальные науки}         %   Факультет
\facultyshort{ФН}                       %   Факультет -- аббревиатура
\chair{Математическое моделирование}    %   кафедра
\chairno{ФН-12}                         %   кафедра -- аббревиатура
\chairhead{А.П. Крищенко}               %   зав. кафедрой
\chief{И.А. Ташков}                     %   Руководитель работы
%\consultant{}               %   Консультант (если нужно)
%\inspector{}                 %   Нормоконтролер
\trend{учебная}                         %   Направление работы (указывается в задании)
\themesource{кафедра}                   %   Источник темы (указывается в задании)

% \equationnumbering{section}             %   если нумерация формул в пределах раздела (иначе единая нумерация)
% \figurenumbering{section}              %   (аналогично для рисунков)
%%%   Оформление содержания, в текст вставлять командой \tableofcontents
\SkipContentsbreak                     %   Подавление разрыва страницы после содержания
%\setcontentsname{Оглавление}           %   Заголовок содержания
\def\contsectionfont{\bfseries}        %   шрифт для разделов в содержании
\def\contsubsectionfont{\sf}           %   шрифт для подразделов в содержании
\setcounter{secnumdepth}{1}            %   глубина заголовков в
